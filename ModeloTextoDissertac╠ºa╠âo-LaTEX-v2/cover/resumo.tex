\chapter*{Resumo}
\addcontentsline{toc}{chapter}{Resumo}

 Este trabalho objetiva contribuir para o conhecimento da Categoria 7 (\textit{"Galaxies with Jets"}) presente no Catálogo Arp \& Madore de Galáxias Peculiares (\textit{"A Catalogue of Southern Peculiar Galaxies and Association, 1987"}). Em particular, os objetos que fazem parte desta Categoria, i.e., as galáxias elípticas (E), semelhantes à elípticas (E-like), espirais (S), de outros tipos (X) e companheiras (C), apresentam um importante processo físico peculiar (jatos) que ainda não está totalmente compreendido para os objetos reunidos nesta Categoria. Portanto, para fomentar esta discussão, reunimos da literatura um conjunto de dados fotométricos (em diversas bandas espectrais) e espectroscópicos (no óptico), onde inferimos algumas correlações de cor e de atividade nuclear para amostra. Dados espectroscópicos de fenda longa foram também obtidos para alguns objetos no OPD/LNA-MCTIC. Porém como as observações originais realizadas por Arp \& Madore foram obtidas com placas fotográficas,  resolvemos pontuar alguns critérios que permitissem garantir se, de fato, os objetos catalogados possuíam jatos ou se eram outras característica que se assemelham a jatos. Ou, ainda, se eram peculiaridades como caudas, laços de matéria e/ou detritos, os quais integram a Categoria 15 do Catálogo. Os 125 espectros finais obtidos, 52 com linhas de absorção e 73 com linhas de emissão, foram analisados com o código STARLIGHT de síntese espectral, o qual forneceu um vetor contendo as populações estelares presentes, assim como as idades e as metalicidades. Apesar destes espectros não conterem informações dos jatos (fracos em sua grande maioria), obtivemos algumas análises preliminares baseadas nos atuais conhecimentos sobre as emissões de jatos relativísticos presentes em radiogaláxias. Como os possíveis processos físicos que podem caracterizar as peculiaridades observadas nos objetos Arp \& Madore são oriundos de interações gravitacionais do tipo colisão, fusão ou maré, acreditamos que os buracos negros assim formados e que habitam o centro destes objetos também representam um mecanismo chave para explicar a presença dos jatos observados, mas com a ressalva que estes acretam o gás na vizinhança a baixas taxas de acreção. Finalmente, com o propósito de manter viva a discussão deste trabalho, uma plataforma Web foi construída com todas as informações e dados desta dissertação. Contribuições fotométricas e espectroscópicas podem ser alimentadas por pesquisadores externos que tenham o interesse em compreender a natureza peculiar desses objetos.

\vspace{.5cm}

\textbf{Palavras-chave:} Galáxias Peculiares. Jatos. Síntese Espectral. Redução de Dados.

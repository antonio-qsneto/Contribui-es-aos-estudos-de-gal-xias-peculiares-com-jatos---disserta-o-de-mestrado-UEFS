%%%%%%%%%%%%%%%%%%%%%%%%%%%%%%%%%%%%%%%%%%%%%%%%%%%%%
%   REVISÃO BIBLIOGRÁFICA
%%%%%%%%%%%%%%%%%%%%%%%%%%%%%%%%%%%%%%%%%%%%%%%%%%%%%

\chapter{Revisao  Bibliográfica}

Antes das galáxias peculiares do Hemisfério Sul serem compiladas em um catálogo geral com diferentes Categorias por \cite{arp1987catalogue}, \textit{“A Catalogue of Southern Peculiar Galaxies and Associations”}, observações no óptico de alguns objetos da Categoria selecionada para este trabalho já haviam sido observados e catalogados previamente por \cite{dreyer}, \cite{vorotsov1968} e \cite{vorontsov1977atlas}.

Com a conclusão final do catálogo por Arp \& Madore, o número de resultados fotométricos e espectroscópicos começou a revelar a verdadeira natureza dessas galáxias. \cite{menzies} realizou um survey no óptico (3700-5400{\AA}) para determinar redshifts precisos (velocidades radiais com desvios padrões da ordem de 35 km/s) para cerca de 500 galáxias, das quais cerca de 10\% da Categoria 7 estavam presentes. Com o mesmo objetivo de compliar um catálogo de redshift para  as galáxias do Hemisfério Sul, \cite{dacosta1991} amplia um pouco mais o trabalho anterior, porém com um erro associado um pouco maior, da ordem de 40 m/s. Segundo os autores, alguns objetos apresentam um baixo brilho superficial, sendo, portanto, muito difícil a determinação das velocidades radiais, ocasionando, desse modo, uma maior incerteza nas medidas.

Um importante catálogo fotométrico de galáxias brilhantes, RC3, foi introdizido por \cite{Vaucouleurs}. Claramente, boa parte dos objetos pertencentes à Categoria 7 (assim como de várias outras), não foram incluídos devido ao baixo brilho superficial. Contudo, algumas importantes informações foram obtidas para alguns objetos e estão presentes em ambas bases de dados usadas neste trabalho, NED/NASA-IPAC e HyperLeda \cite{paturel2003hyperleda}.

O survey no óptico realizado por \cite{dacosta1991} foi ampliado em \cite{dacosta1998}, com a introdução de magnitudes e classificações morfológicas para uma amostra de 5369 galáxias com m$_{B}$ $\leq$ 15,5. Os objetos tiveram posições precisas de ~1” e magnitudes com um “rms” da ordem de 0,3 mag. Apesar do esforço realizado, a grande maioria dos nossos objetos de interesse ficaram fora do estudo.

Um resultado interessante foi publicado por \cite{donzelli2000spectroscopic}, onde os autores discutem as propriedades espectroscópicas no óptico de galáxias em interação (“merging”). A partir desta publicação, iremos extrair para aqueles objetos onde não obtemos dados de fenda longa no OPD/LNA, informações que poderão contribuir para entender a natureza morfológica dos jatos observados. Uma fusão de galáxias, envolvendo a Física dos buracos negros envolvidos, representa uma excelente pista para a compreensão deste particular fenômeno.

Uma contribuição no infravermelho próximo foi apresentada por \cite{monnierb,monniera}, através do mapeamento da brilho superficial para cerca de 3000 galáxias nas bandas J, H e Ks (2MASS). Uma análise global ainda está sendo feita e poderá fornecer importantes pistas que poderão complementar a análise visual de alguns objetos da nossa amostra. O estudo do gás em 1038 galáxias interagentes foi apresetada por \cite{casasola2004gas}. Um resultado interessante parte da análise de que o excesso de gás molecular detectado pode ser derivado de efeitos de torque (maré). Assim, além das fusões observadas, as interações por maré sofridas por algumas galáxias das Categorias analisadas no artigo, também serão extraídas para serem estudadas de forma mais criteriosa, como um desdobramento deste trabalho.

Um survey de imageamento nas linhas do H$\alpha$ + [N II] foi produzido por \cite{kennicutt08}, como um esforço para caracterizar a população estelar de galáxias no universo local. Esta análise inferi diretamente no tipo particular de atividade nuclear, que representa um dos pontos a ser atacado neste trabalho através da classificação espectral. Logicamente, os resultados serão confrontados e deverão ampliar a nossa base de objetos da Categoria 7 com formação estelar. Em adiçao, os resutlados também apresentados por \cite{karachentsev2013star} sobre as propriedades de formação estelar (H$\alpha$ + ultravioleta distante-GALEX) serão analisados a luz da taxa de formação estelar esperada para uma distância de 11Mpc. Alguns dos nossos objetos de estudo encontram-se dentro desta estimativa de distância.

Fotometria de multi-banda profunda de solo (UBVRIHKs) no óptico e no infravermelho próximo \cite{micheva2013deep}, assim como dados no óptico obtido com o satélite Spitzer \cite{knapp2014interpersonal}, serão explorados afim de aumentar o número de informações destes objetos. Dados também fornecidos por este satélite referente a classificação morfológica \cite{buta2015classical} e sobre a estrutura estelar \cite{salo2015}, também serão analisado neste trabalho.

A revisão feita acima é extremamente necessária para fundamentar o objetivo principal a ser obtido neste trabalho. É claro que a mesma não está completa e deverá ser continuamente ampliada, pois alguns artigos interessantes (mais recentes) já foram sinalizados no arXiv-eprint, onde já estamos processando uma análise.
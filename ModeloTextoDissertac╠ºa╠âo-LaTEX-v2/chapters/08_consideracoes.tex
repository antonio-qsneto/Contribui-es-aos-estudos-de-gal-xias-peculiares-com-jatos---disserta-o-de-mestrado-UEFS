%%%%%%%%%%%%%%%%%%%%%%%%%%%%%%%%%%%%%%%%%%%%%%%%%%%%%
%   CONCLUSÃO
%%%%%%%%%%%%%%%%%%%%%%%%%%%%%%%%%%%%%%%%%%%%%%%%%%%%%

\chapter{Considerações Finais}
%\chapter*[Conclusão]{Conclusão} % se quiser que o cap. não seja numerado
%\addcontentsline{toc}{chapter}{Considerações Finais}

O presente estudo objetiva contribuir para o conhecimento das galáxias peculiares que apresentam jatos no óptico. Como são objeto pouco estudados, tanto na fotometria, quanto na espectroscopia, realizamos um levantamento em bancos de dados para fomentar essa discussão. As bandas obtidas revelaram importantes informações sobre as regiões nuclear e extranuclear desses objetos, permitindo correlacionar diretamente, no caso da banda GALEX (ultravioleta próximo), com uma atividade nuclear. Esse aspecto é muito importante, pois permite que tenhamos uma estimativa sobre quantitativo de galáxias ativas da Categoria 7 presentes no Universo local. Neste estudo preliminar, 41,6\% podem ser rotuladas como galáxias normais, sendo a grande maioria da população estelar velha. Por outro lado, a amostragem também revelou que 58,4\% possuem algum tipo de atividade nuclear, que perrmeia desde Starburst (ou regiões HII) até AGN, com populações com representações de estrelas muito jovens, jovens e de idades intermediárias.

Para os espectros, conseguimos reunir uma razoável amostragem no survey conduzido pelo 6dFGS, mas apenas com informações da região nuclear. Isso já era esperado, pois, na grande maioria dos surveys, os levantamentos realizados do céu não são projetos dedicados para explorar um ou outro objeto. Além do mais, o tempo considerado para cada fonte não é suficiente para obter alguma informação espectral do jato.

As observações conduzidas no OPD/LNA-MCTI, dedicadas para o estudo espectroscópicos desses objetos, permitiu de certa forma controlar a síntese espectral nos dados do 6dFGS. Infelizmente, apenas cinco objetos estavam presentes em ambas amostragens. Contudo, os resultados foram satisfatórios, pois as mesmas informações relacionadas como a populações estelares foram obtidas. Isso representa um ponto crítico, pois os dados espectrais do 6dFGS já estavam reduzidos e calibrados.

Embora tenhamos apresentados alguns resultados preliminares para esta particular categoria de objetos peculiares, uma interpretação sobre a origem, colimação, propagação e radiação, ainda encontra-se aberta. Na verdade, trata-se de um problema onde modelos teóricos analíticos e computacionais ainda buscam uma interpretação mais consistente.

Dentro do campo da Astrofı́sica, os buracos negros supermassivos que habitam o centro da maior parte das galáxias na época presente, representam o mecanismo chave para explicar tais fenômenos físicos. No caso particular dos objetos estudados no óptico, podemos inferir que estes também estão presentes, mas acretam gás a baixas taxas de acreção, revelando características relativamente diferentes, sobretudo, em relação as distâncias cobertas de centenas de kiloparsecs a partir da fonte emissora.

Outro aspecto importante sobre os buracos negros, reside no fato que estes dão origem à população dominante de AGN nas 
galáxias próximas. No caso particular dos nossos objetos, os AGN revelados são de baixa luminosidade (LLAGNs), compatíveis 
com aqueles observados no Universo local.

Esperamos ainda explorar os dados obtidos até aqui de outras maneiras, correlacionado, por exemplo, com alguns parâmetros geométricos como a inclinação, razão axial, excentricidade e elipticidade, na perspectiva de inferir se existe alguma correlação com a direção observada do jato. Outra perspectiva está associada as intensidades das linhas de emissão detectadas após a subtração da população estelar subjacente, com o correspondente tipo morfológico. Um número considerável são galáxias com características espirais. Como a emissão desse tipo de galáxia sofre variabilidade em períodos relativamente curtos, podemos concluir que a fonte emissora  deve ser compacta, o que corrobora com o cenário descrito acima, que estas galáxia acretam gás a baixas taxas de acreção, quando comparadas as radiogaláxias.

Finalmente, a contribuição computacional dada através da implantação do sistema Web permitirá que diversos pesquisadores possam alimentar com dados e informações que permitam compreender a natureza peculiar da Categoria 7.

Alguns exemplos de galáxias com linhas de absorção e emissão, além da síntese espectral, são mostradas nos Apêndices A, B e C, respectivamente. Todas os objetos estarão disponíveis para consulta interativa no sistema Web.
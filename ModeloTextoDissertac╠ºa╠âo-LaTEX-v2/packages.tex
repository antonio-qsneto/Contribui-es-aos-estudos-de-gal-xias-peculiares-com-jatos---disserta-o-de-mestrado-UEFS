\sloppy

\usepackage{pdfpages}
\usepackage{verbatim}

%%%%%%%%%%%%%%%%%%%%%%%%%%
\usepackage{cmap}				% Mapear caracteres especiais no PDF
\usepackage{lmodern}			% Usa a fonte Latin Modern			
\usepackage[T1]{fontenc}		% Selecao de codigos de fonte.
\usepackage[utf8]{inputenc}		% Codificacao do documento (conversão automática dos acentos)
\usepackage{lastpage}			% Usado pela Ficha catalográfica
\usepackage{indentfirst}		% Indenta o primeiro parágrafo de cada seção.
\usepackage{color}				% Controle das cores
\usepackage{graphicx}			% Inclusão de gráficos
\usepackage{rotating}          % usar tabela na página em paisagem
\usepackage{longtable}          % para usar tabelas longas
% ---
% For Code
\usepackage{listings}
\usepackage{minted}
% ---
% Pacotes adicionais, usados apenas no âmbito do Modelo Canônico do abnteX2
% ---
\usepackage{lipsum}				% para geração de dummy text
% ---

\usepackage[brazilian,hyperpageref]{backref}	 % Paginas com as citações na bibl
%\usepackage[alf]{abntex2cite}	% Citações padrão ABNT
%%%%%%%%%%%%%%%%%%%%%%%%%%
\usepackage{xcolor}
% Definindo novas cores
\definecolor{verde}{rgb}{0,0.5,0}
% Configurando layout para mostrar codigos C++
\usepackage{listings}
\lstset{
  language=php,
  basicstyle=\ttfamily\small, 
  keywordstyle=\color{blue}, 
  stringstyle=\color{verde}, 
  commentstyle=\color{gray}, 
  extendedchars=true, 
  showspaces=false, 
  showstringspaces=false, 
  numbers=left,
  numberstyle=\tiny,
  breaklines=true, 
  backgroundcolor=\color{black!10},
  breakautoindent=true, 
  captionpos=b,
  xleftmargin=0pt,
}
%%%%%%%%%%%%%%%%%%%%%%%%%%
\usepackage[T1]{fontenc} %write accents
\usepackage[latin1]{inputenc} %permits to write the text with accents
\usepackage[portuguese]{babel} %put dates in portuguese
\usepackage{changepage} %Margin adjustment and detection of odd/even pages
%%%%%%%%%%%%%%%%
\usepackage{float}
%%%%%%%%%%%%%%%%
%To prepare list of symbols
\usepackage{longtable}

% Sets the margins of the document.
\oddsidemargin 5mm
\evensidemargin 5mm
\textwidth 150mm
\topmargin 0mm
\headheight 0mm
\textheight 225mm

% Selects font encoding
\usepackage[T1]{fontenc}
\usepackage{ae,aecompl}
%\usepackage{times}

\usepackage{lipsum}

\usepackage{epigraph}
\setlength{\epigraphrule}{0pt}
\setlength{\afterepigraphskip}{2\baselineskip}

%header displays information according to document class and page number top right.
\usepackage{fancyhdr}


% Starts new paragraphs without indentation but with some space between the new and the previous paragraph.
\usepackage{parskip}
\setlength{\parskip}{1.5ex plus 0.4ex minus 0.4ex}
%\usepackage{indentfirst}

% try to keep paragraphs together
\widowpenalty=300
\clubpenalty=300

% Used to create tables with rows/cols spanning over se
\usepackage{array}
\newcolumntype{L}[1]{>{\arraybackslash}p{#1cm}}
\newcolumntype{C}[1]{>{\centering\arraybackslash}p{#1cm}}
\usepackage{multirow}

%Professional tables
\usepackage{booktabs}

% Math
\usepackage{amssymb}
\usepackage{amsmath}

% Used to include images with the includegraphics command
\usepackage{epsfig,subfigure,amstext}
\usepackage{graphicx}

\usepackage{makeidx}
\makeindex

%permits relative paths in the imported files
\usepackage{import}

%permits to create different environments
\usepackage{amsthm}

\newtheoremstyle{simple}
   {8pt}% hSpace above
   {}% hSpace below
   {\it}% hBody font
   {1em}% hIndent amount
   {}% hTheorem head font
   {\textup{:}}% hPunctuation after theorem head
   {.7em}% hSpace after theorem head
   {}% hTheorem head spec (can be left empty, meaning `normal')

\newtheoremstyle{enhanced}
   {8pt}% hSpace above
   {}% hSpace below
   {}% hBody font
   {1em}% hIndent amount
   {\itshape}% hTheorem head font
   {\textup{:}}% hPunctuation after theorem head
   {.7em}% hSpace after theorem head
   {}% hTheorem head spec (can be left empty, meaning `normal')


%links in pdf (index, references, and figures..., change the colors to black)
\usepackage[pdfborder={0 0 0},breaklinks=true,bookmarksopen=true,bookmarksopenlevel=1]{hyperref}


% Used to include algorithms
%\usepackage{algorithm,algorithmic}
%\renewcommand{\algorithmiccomment}[1]{\hfill \textit{//#1}}
%\newcommand{\vect}[1]{\overrightarrow{#1}}